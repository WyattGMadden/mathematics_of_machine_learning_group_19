
% Default to the notebook output style

    


% Inherit from the specified cell style.




    
\documentclass[11pt]{article}

    
    
    \usepackage[T1]{fontenc}
    % Nicer default font (+ math font) than Computer Modern for most use cases
    \usepackage{mathpazo}

    % Basic figure setup, for now with no caption control since it's done
    % automatically by Pandoc (which extracts ![](path) syntax from Markdown).
    \usepackage{graphicx}
    % We will generate all images so they have a width \maxwidth. This means
    % that they will get their normal width if they fit onto the page, but
    % are scaled down if they would overflow the margins.
    \makeatletter
    \def\maxwidth{\ifdim\Gin@nat@width>\linewidth\linewidth
    \else\Gin@nat@width\fi}
    \makeatother
    \let\Oldincludegraphics\includegraphics
    % Set max figure width to be 80% of text width, for now hardcoded.
    \renewcommand{\includegraphics}[1]{\Oldincludegraphics[width=.8\maxwidth]{#1}}
    % Ensure that by default, figures have no caption (until we provide a
    % proper Figure object with a Caption API and a way to capture that
    % in the conversion process - todo).
    \usepackage{caption}
    \DeclareCaptionLabelFormat{nolabel}{}
    \captionsetup{labelformat=nolabel}

    \usepackage{adjustbox} % Used to constrain images to a maximum size 
    \usepackage{xcolor} % Allow colors to be defined
    \usepackage{enumerate} % Needed for markdown enumerations to work
    \usepackage{geometry} % Used to adjust the document margins
    \usepackage{amsmath} % Equations
    \usepackage{amssymb} % Equations
    \usepackage{textcomp} % defines textquotesingle
    % Hack from http://tex.stackexchange.com/a/47451/13684:
    \AtBeginDocument{%
        \def\PYZsq{\textquotesingle}% Upright quotes in Pygmentized code
    }
    \usepackage{upquote} % Upright quotes for verbatim code
    \usepackage{eurosym} % defines \euro
    \usepackage[mathletters]{ucs} % Extended unicode (utf-8) support
    \usepackage[utf8x]{inputenc} % Allow utf-8 characters in the tex document
    \usepackage{fancyvrb} % verbatim replacement that allows latex
    \usepackage{grffile} % extends the file name processing of package graphics 
                         % to support a larger range 
    % The hyperref package gives us a pdf with properly built
    % internal navigation ('pdf bookmarks' for the table of contents,
    % internal cross-reference links, web links for URLs, etc.)
    \usepackage{hyperref}
    \usepackage{longtable} % longtable support required by pandoc >1.10
    \usepackage{booktabs}  % table support for pandoc > 1.12.2
    \usepackage[inline]{enumitem} % IRkernel/repr support (it uses the enumerate* environment)
    \usepackage[normalem]{ulem} % ulem is needed to support strikethroughs (\sout)
                                % normalem makes italics be italics, not underlines
    

    
    
    % Colors for the hyperref package
    \definecolor{urlcolor}{rgb}{0,.145,.698}
    \definecolor{linkcolor}{rgb}{.71,0.21,0.01}
    \definecolor{citecolor}{rgb}{.12,.54,.11}

    % ANSI colors
    \definecolor{ansi-black}{HTML}{3E424D}
    \definecolor{ansi-black-intense}{HTML}{282C36}
    \definecolor{ansi-red}{HTML}{E75C58}
    \definecolor{ansi-red-intense}{HTML}{B22B31}
    \definecolor{ansi-green}{HTML}{00A250}
    \definecolor{ansi-green-intense}{HTML}{007427}
    \definecolor{ansi-yellow}{HTML}{DDB62B}
    \definecolor{ansi-yellow-intense}{HTML}{B27D12}
    \definecolor{ansi-blue}{HTML}{208FFB}
    \definecolor{ansi-blue-intense}{HTML}{0065CA}
    \definecolor{ansi-magenta}{HTML}{D160C4}
    \definecolor{ansi-magenta-intense}{HTML}{A03196}
    \definecolor{ansi-cyan}{HTML}{60C6C8}
    \definecolor{ansi-cyan-intense}{HTML}{258F8F}
    \definecolor{ansi-white}{HTML}{C5C1B4}
    \definecolor{ansi-white-intense}{HTML}{A1A6B2}

    % commands and environments needed by pandoc snippets
    % extracted from the output of `pandoc -s`
    \providecommand{\tightlist}{%
      \setlength{\itemsep}{0pt}\setlength{\parskip}{0pt}}
    \DefineVerbatimEnvironment{Highlighting}{Verbatim}{commandchars=\\\{\}}
    % Add ',fontsize=\small' for more characters per line
    \newenvironment{Shaded}{}{}
    \newcommand{\KeywordTok}[1]{\textcolor[rgb]{0.00,0.44,0.13}{\textbf{{#1}}}}
    \newcommand{\DataTypeTok}[1]{\textcolor[rgb]{0.56,0.13,0.00}{{#1}}}
    \newcommand{\DecValTok}[1]{\textcolor[rgb]{0.25,0.63,0.44}{{#1}}}
    \newcommand{\BaseNTok}[1]{\textcolor[rgb]{0.25,0.63,0.44}{{#1}}}
    \newcommand{\FloatTok}[1]{\textcolor[rgb]{0.25,0.63,0.44}{{#1}}}
    \newcommand{\CharTok}[1]{\textcolor[rgb]{0.25,0.44,0.63}{{#1}}}
    \newcommand{\StringTok}[1]{\textcolor[rgb]{0.25,0.44,0.63}{{#1}}}
    \newcommand{\CommentTok}[1]{\textcolor[rgb]{0.38,0.63,0.69}{\textit{{#1}}}}
    \newcommand{\OtherTok}[1]{\textcolor[rgb]{0.00,0.44,0.13}{{#1}}}
    \newcommand{\AlertTok}[1]{\textcolor[rgb]{1.00,0.00,0.00}{\textbf{{#1}}}}
    \newcommand{\FunctionTok}[1]{\textcolor[rgb]{0.02,0.16,0.49}{{#1}}}
    \newcommand{\RegionMarkerTok}[1]{{#1}}
    \newcommand{\ErrorTok}[1]{\textcolor[rgb]{1.00,0.00,0.00}{\textbf{{#1}}}}
    \newcommand{\NormalTok}[1]{{#1}}
    
    % Additional commands for more recent versions of Pandoc
    \newcommand{\ConstantTok}[1]{\textcolor[rgb]{0.53,0.00,0.00}{{#1}}}
    \newcommand{\SpecialCharTok}[1]{\textcolor[rgb]{0.25,0.44,0.63}{{#1}}}
    \newcommand{\VerbatimStringTok}[1]{\textcolor[rgb]{0.25,0.44,0.63}{{#1}}}
    \newcommand{\SpecialStringTok}[1]{\textcolor[rgb]{0.73,0.40,0.53}{{#1}}}
    \newcommand{\ImportTok}[1]{{#1}}
    \newcommand{\DocumentationTok}[1]{\textcolor[rgb]{0.73,0.13,0.13}{\textit{{#1}}}}
    \newcommand{\AnnotationTok}[1]{\textcolor[rgb]{0.38,0.63,0.69}{\textbf{\textit{{#1}}}}}
    \newcommand{\CommentVarTok}[1]{\textcolor[rgb]{0.38,0.63,0.69}{\textbf{\textit{{#1}}}}}
    \newcommand{\VariableTok}[1]{\textcolor[rgb]{0.10,0.09,0.49}{{#1}}}
    \newcommand{\ControlFlowTok}[1]{\textcolor[rgb]{0.00,0.44,0.13}{\textbf{{#1}}}}
    \newcommand{\OperatorTok}[1]{\textcolor[rgb]{0.40,0.40,0.40}{{#1}}}
    \newcommand{\BuiltInTok}[1]{{#1}}
    \newcommand{\ExtensionTok}[1]{{#1}}
    \newcommand{\PreprocessorTok}[1]{\textcolor[rgb]{0.74,0.48,0.00}{{#1}}}
    \newcommand{\AttributeTok}[1]{\textcolor[rgb]{0.49,0.56,0.16}{{#1}}}
    \newcommand{\InformationTok}[1]{\textcolor[rgb]{0.38,0.63,0.69}{\textbf{\textit{{#1}}}}}
    \newcommand{\WarningTok}[1]{\textcolor[rgb]{0.38,0.63,0.69}{\textbf{\textit{{#1}}}}}
    
    
    % Define a nice break command that doesn't care if a line doesn't already
    % exist.
    \def\br{\hspace*{\fill} \\* }
    % Math Jax compatability definitions
    \def\gt{>}
    \def\lt{<}
    % Document parameters
    \title{Lab 4 - Wyatt Madden \& Dan Crowley}
    
    
    

    % Pygments definitions
    
\makeatletter
\def\PY@reset{\let\PY@it=\relax \let\PY@bf=\relax%
    \let\PY@ul=\relax \let\PY@tc=\relax%
    \let\PY@bc=\relax \let\PY@ff=\relax}
\def\PY@tok#1{\csname PY@tok@#1\endcsname}
\def\PY@toks#1+{\ifx\relax#1\empty\else%
    \PY@tok{#1}\expandafter\PY@toks\fi}
\def\PY@do#1{\PY@bc{\PY@tc{\PY@ul{%
    \PY@it{\PY@bf{\PY@ff{#1}}}}}}}
\def\PY#1#2{\PY@reset\PY@toks#1+\relax+\PY@do{#2}}

\expandafter\def\csname PY@tok@w\endcsname{\def\PY@tc##1{\textcolor[rgb]{0.73,0.73,0.73}{##1}}}
\expandafter\def\csname PY@tok@c\endcsname{\let\PY@it=\textit\def\PY@tc##1{\textcolor[rgb]{0.25,0.50,0.50}{##1}}}
\expandafter\def\csname PY@tok@cp\endcsname{\def\PY@tc##1{\textcolor[rgb]{0.74,0.48,0.00}{##1}}}
\expandafter\def\csname PY@tok@k\endcsname{\let\PY@bf=\textbf\def\PY@tc##1{\textcolor[rgb]{0.00,0.50,0.00}{##1}}}
\expandafter\def\csname PY@tok@kp\endcsname{\def\PY@tc##1{\textcolor[rgb]{0.00,0.50,0.00}{##1}}}
\expandafter\def\csname PY@tok@kt\endcsname{\def\PY@tc##1{\textcolor[rgb]{0.69,0.00,0.25}{##1}}}
\expandafter\def\csname PY@tok@o\endcsname{\def\PY@tc##1{\textcolor[rgb]{0.40,0.40,0.40}{##1}}}
\expandafter\def\csname PY@tok@ow\endcsname{\let\PY@bf=\textbf\def\PY@tc##1{\textcolor[rgb]{0.67,0.13,1.00}{##1}}}
\expandafter\def\csname PY@tok@nb\endcsname{\def\PY@tc##1{\textcolor[rgb]{0.00,0.50,0.00}{##1}}}
\expandafter\def\csname PY@tok@nf\endcsname{\def\PY@tc##1{\textcolor[rgb]{0.00,0.00,1.00}{##1}}}
\expandafter\def\csname PY@tok@nc\endcsname{\let\PY@bf=\textbf\def\PY@tc##1{\textcolor[rgb]{0.00,0.00,1.00}{##1}}}
\expandafter\def\csname PY@tok@nn\endcsname{\let\PY@bf=\textbf\def\PY@tc##1{\textcolor[rgb]{0.00,0.00,1.00}{##1}}}
\expandafter\def\csname PY@tok@ne\endcsname{\let\PY@bf=\textbf\def\PY@tc##1{\textcolor[rgb]{0.82,0.25,0.23}{##1}}}
\expandafter\def\csname PY@tok@nv\endcsname{\def\PY@tc##1{\textcolor[rgb]{0.10,0.09,0.49}{##1}}}
\expandafter\def\csname PY@tok@no\endcsname{\def\PY@tc##1{\textcolor[rgb]{0.53,0.00,0.00}{##1}}}
\expandafter\def\csname PY@tok@nl\endcsname{\def\PY@tc##1{\textcolor[rgb]{0.63,0.63,0.00}{##1}}}
\expandafter\def\csname PY@tok@ni\endcsname{\let\PY@bf=\textbf\def\PY@tc##1{\textcolor[rgb]{0.60,0.60,0.60}{##1}}}
\expandafter\def\csname PY@tok@na\endcsname{\def\PY@tc##1{\textcolor[rgb]{0.49,0.56,0.16}{##1}}}
\expandafter\def\csname PY@tok@nt\endcsname{\let\PY@bf=\textbf\def\PY@tc##1{\textcolor[rgb]{0.00,0.50,0.00}{##1}}}
\expandafter\def\csname PY@tok@nd\endcsname{\def\PY@tc##1{\textcolor[rgb]{0.67,0.13,1.00}{##1}}}
\expandafter\def\csname PY@tok@s\endcsname{\def\PY@tc##1{\textcolor[rgb]{0.73,0.13,0.13}{##1}}}
\expandafter\def\csname PY@tok@sd\endcsname{\let\PY@it=\textit\def\PY@tc##1{\textcolor[rgb]{0.73,0.13,0.13}{##1}}}
\expandafter\def\csname PY@tok@si\endcsname{\let\PY@bf=\textbf\def\PY@tc##1{\textcolor[rgb]{0.73,0.40,0.53}{##1}}}
\expandafter\def\csname PY@tok@se\endcsname{\let\PY@bf=\textbf\def\PY@tc##1{\textcolor[rgb]{0.73,0.40,0.13}{##1}}}
\expandafter\def\csname PY@tok@sr\endcsname{\def\PY@tc##1{\textcolor[rgb]{0.73,0.40,0.53}{##1}}}
\expandafter\def\csname PY@tok@ss\endcsname{\def\PY@tc##1{\textcolor[rgb]{0.10,0.09,0.49}{##1}}}
\expandafter\def\csname PY@tok@sx\endcsname{\def\PY@tc##1{\textcolor[rgb]{0.00,0.50,0.00}{##1}}}
\expandafter\def\csname PY@tok@m\endcsname{\def\PY@tc##1{\textcolor[rgb]{0.40,0.40,0.40}{##1}}}
\expandafter\def\csname PY@tok@gh\endcsname{\let\PY@bf=\textbf\def\PY@tc##1{\textcolor[rgb]{0.00,0.00,0.50}{##1}}}
\expandafter\def\csname PY@tok@gu\endcsname{\let\PY@bf=\textbf\def\PY@tc##1{\textcolor[rgb]{0.50,0.00,0.50}{##1}}}
\expandafter\def\csname PY@tok@gd\endcsname{\def\PY@tc##1{\textcolor[rgb]{0.63,0.00,0.00}{##1}}}
\expandafter\def\csname PY@tok@gi\endcsname{\def\PY@tc##1{\textcolor[rgb]{0.00,0.63,0.00}{##1}}}
\expandafter\def\csname PY@tok@gr\endcsname{\def\PY@tc##1{\textcolor[rgb]{1.00,0.00,0.00}{##1}}}
\expandafter\def\csname PY@tok@ge\endcsname{\let\PY@it=\textit}
\expandafter\def\csname PY@tok@gs\endcsname{\let\PY@bf=\textbf}
\expandafter\def\csname PY@tok@gp\endcsname{\let\PY@bf=\textbf\def\PY@tc##1{\textcolor[rgb]{0.00,0.00,0.50}{##1}}}
\expandafter\def\csname PY@tok@go\endcsname{\def\PY@tc##1{\textcolor[rgb]{0.53,0.53,0.53}{##1}}}
\expandafter\def\csname PY@tok@gt\endcsname{\def\PY@tc##1{\textcolor[rgb]{0.00,0.27,0.87}{##1}}}
\expandafter\def\csname PY@tok@err\endcsname{\def\PY@bc##1{\setlength{\fboxsep}{0pt}\fcolorbox[rgb]{1.00,0.00,0.00}{1,1,1}{\strut ##1}}}
\expandafter\def\csname PY@tok@kc\endcsname{\let\PY@bf=\textbf\def\PY@tc##1{\textcolor[rgb]{0.00,0.50,0.00}{##1}}}
\expandafter\def\csname PY@tok@kd\endcsname{\let\PY@bf=\textbf\def\PY@tc##1{\textcolor[rgb]{0.00,0.50,0.00}{##1}}}
\expandafter\def\csname PY@tok@kn\endcsname{\let\PY@bf=\textbf\def\PY@tc##1{\textcolor[rgb]{0.00,0.50,0.00}{##1}}}
\expandafter\def\csname PY@tok@kr\endcsname{\let\PY@bf=\textbf\def\PY@tc##1{\textcolor[rgb]{0.00,0.50,0.00}{##1}}}
\expandafter\def\csname PY@tok@bp\endcsname{\def\PY@tc##1{\textcolor[rgb]{0.00,0.50,0.00}{##1}}}
\expandafter\def\csname PY@tok@fm\endcsname{\def\PY@tc##1{\textcolor[rgb]{0.00,0.00,1.00}{##1}}}
\expandafter\def\csname PY@tok@vc\endcsname{\def\PY@tc##1{\textcolor[rgb]{0.10,0.09,0.49}{##1}}}
\expandafter\def\csname PY@tok@vg\endcsname{\def\PY@tc##1{\textcolor[rgb]{0.10,0.09,0.49}{##1}}}
\expandafter\def\csname PY@tok@vi\endcsname{\def\PY@tc##1{\textcolor[rgb]{0.10,0.09,0.49}{##1}}}
\expandafter\def\csname PY@tok@vm\endcsname{\def\PY@tc##1{\textcolor[rgb]{0.10,0.09,0.49}{##1}}}
\expandafter\def\csname PY@tok@sa\endcsname{\def\PY@tc##1{\textcolor[rgb]{0.73,0.13,0.13}{##1}}}
\expandafter\def\csname PY@tok@sb\endcsname{\def\PY@tc##1{\textcolor[rgb]{0.73,0.13,0.13}{##1}}}
\expandafter\def\csname PY@tok@sc\endcsname{\def\PY@tc##1{\textcolor[rgb]{0.73,0.13,0.13}{##1}}}
\expandafter\def\csname PY@tok@dl\endcsname{\def\PY@tc##1{\textcolor[rgb]{0.73,0.13,0.13}{##1}}}
\expandafter\def\csname PY@tok@s2\endcsname{\def\PY@tc##1{\textcolor[rgb]{0.73,0.13,0.13}{##1}}}
\expandafter\def\csname PY@tok@sh\endcsname{\def\PY@tc##1{\textcolor[rgb]{0.73,0.13,0.13}{##1}}}
\expandafter\def\csname PY@tok@s1\endcsname{\def\PY@tc##1{\textcolor[rgb]{0.73,0.13,0.13}{##1}}}
\expandafter\def\csname PY@tok@mb\endcsname{\def\PY@tc##1{\textcolor[rgb]{0.40,0.40,0.40}{##1}}}
\expandafter\def\csname PY@tok@mf\endcsname{\def\PY@tc##1{\textcolor[rgb]{0.40,0.40,0.40}{##1}}}
\expandafter\def\csname PY@tok@mh\endcsname{\def\PY@tc##1{\textcolor[rgb]{0.40,0.40,0.40}{##1}}}
\expandafter\def\csname PY@tok@mi\endcsname{\def\PY@tc##1{\textcolor[rgb]{0.40,0.40,0.40}{##1}}}
\expandafter\def\csname PY@tok@il\endcsname{\def\PY@tc##1{\textcolor[rgb]{0.40,0.40,0.40}{##1}}}
\expandafter\def\csname PY@tok@mo\endcsname{\def\PY@tc##1{\textcolor[rgb]{0.40,0.40,0.40}{##1}}}
\expandafter\def\csname PY@tok@ch\endcsname{\let\PY@it=\textit\def\PY@tc##1{\textcolor[rgb]{0.25,0.50,0.50}{##1}}}
\expandafter\def\csname PY@tok@cm\endcsname{\let\PY@it=\textit\def\PY@tc##1{\textcolor[rgb]{0.25,0.50,0.50}{##1}}}
\expandafter\def\csname PY@tok@cpf\endcsname{\let\PY@it=\textit\def\PY@tc##1{\textcolor[rgb]{0.25,0.50,0.50}{##1}}}
\expandafter\def\csname PY@tok@c1\endcsname{\let\PY@it=\textit\def\PY@tc##1{\textcolor[rgb]{0.25,0.50,0.50}{##1}}}
\expandafter\def\csname PY@tok@cs\endcsname{\let\PY@it=\textit\def\PY@tc##1{\textcolor[rgb]{0.25,0.50,0.50}{##1}}}

\def\PYZbs{\char`\\}
\def\PYZus{\char`\_}
\def\PYZob{\char`\{}
\def\PYZcb{\char`\}}
\def\PYZca{\char`\^}
\def\PYZam{\char`\&}
\def\PYZlt{\char`\<}
\def\PYZgt{\char`\>}
\def\PYZsh{\char`\#}
\def\PYZpc{\char`\%}
\def\PYZdl{\char`\$}
\def\PYZhy{\char`\-}
\def\PYZsq{\char`\'}
\def\PYZdq{\char`\"}
\def\PYZti{\char`\~}
% for compatibility with earlier versions
\def\PYZat{@}
\def\PYZlb{[}
\def\PYZrb{]}
\makeatother


    % Exact colors from NB
    \definecolor{incolor}{rgb}{0.0, 0.0, 0.5}
    \definecolor{outcolor}{rgb}{0.545, 0.0, 0.0}



    
    % Prevent overflowing lines due to hard-to-break entities
    \sloppy 
    % Setup hyperref package
    \hypersetup{
      breaklinks=true,  % so long urls are correctly broken across lines
      colorlinks=true,
      urlcolor=urlcolor,
      linkcolor=linkcolor,
      citecolor=citecolor,
      }
    % Slightly bigger margins than the latex defaults
    
    \geometry{verbose,tmargin=1in,bmargin=1in,lmargin=1in,rmargin=1in}
    
    

    \begin{document}
    
    
    \maketitle
    
    

    
    \section{3.1}\label{section}

    \begin{Verbatim}[commandchars=\\\{\}]
{\color{incolor}In [{\color{incolor}1}]:} \PY{k+kn}{import} \PY{n+nn}{scipy}\PY{n+nn}{.}\PY{n+nn}{io} \PY{k}{as} \PY{n+nn}{scipy}
        \PY{k+kn}{import} \PY{n+nn}{seaborn} \PY{k}{as} \PY{n+nn}{sns}
        \PY{k+kn}{import} \PY{n+nn}{matplotlib}\PY{n+nn}{.}\PY{n+nn}{pyplot} \PY{k}{as} \PY{n+nn}{plt}
        \PY{k+kn}{import} \PY{n+nn}{pandas} \PY{k}{as} \PY{n+nn}{pd}
        \PY{k+kn}{import} \PY{n+nn}{numpy} \PY{k}{as} \PY{n+nn}{np}
        \PY{k+kn}{import} \PY{n+nn}{numpy}\PY{n+nn}{.}\PY{n+nn}{linalg} \PY{k}{as} \PY{n+nn}{lg}
        \PY{k+kn}{from} \PY{n+nn}{eval\PYZus{}basis} \PY{k}{import} \PY{o}{*}
        \PY{k+kn}{from} \PY{n+nn}{func\PYZus{}gauss} \PY{k}{import} \PY{o}{*}
        \PY{k+kn}{from} \PY{n+nn}{func\PYZus{}hat} \PY{k}{import} \PY{o}{*}
        \PY{k+kn}{from} \PY{n+nn}{gauss\PYZus{}basis} \PY{k}{import} \PY{o}{*}
        \PY{k+kn}{from} \PY{n+nn}{hat\PYZus{}basis} \PY{k}{import} \PY{o}{*}
\end{Verbatim}


    \begin{Verbatim}[commandchars=\\\{\}]
{\color{incolor}In [{\color{incolor}2}]:} \PY{c+c1}{\PYZsh{} Least\PYZhy{}Squared Error FIT}
        \PY{c+c1}{\PYZsh{}  Find the linear combination of basis functions which best model the data.}
        \PY{c+c1}{\PYZsh{}}
        \PY{c+c1}{\PYZsh{}  Inputs:}
        \PY{c+c1}{\PYZsh{}  }
        \PY{c+c1}{\PYZsh{}  x \PYZhy{} Vector with observation locations in 1D. (indep. variable)}
        \PY{c+c1}{\PYZsh{}  t \PYZhy{} Vector with observations in 1D. (dep. variable)}
        \PY{c+c1}{\PYZsh{}  params \PYZhy{} Parameters for the basis functions to be used in func, e.g. as}
        \PY{c+c1}{\PYZsh{}    produced by gauss\PYZus{}basis.}
        \PY{c+c1}{\PYZsh{}  func \PYZhy{} Function handle which evaluates a basis function with parameters}
        \PY{c+c1}{\PYZsh{}    given by the columns of params and at the specified locations. e.g. }
        \PY{c+c1}{\PYZsh{}    @gauss\PYZus{}basis, or @hat\PYZus{}basis.}
        \PY{c+c1}{\PYZsh{}    For example, the first basis function at x = 2 is func(2, params(:,1)).}
        \PY{c+c1}{\PYZsh{}  mu \PYZhy{} Scalar representing the standard deviation of the prior Gaussian on}
        \PY{c+c1}{\PYZsh{}    the model parameters.}
        \PY{c+c1}{\PYZsh{}}
        \PY{c+c1}{\PYZsh{}  Outputs:}
        \PY{c+c1}{\PYZsh{}  w \PYZhy{} Coefficients used to generate a linear combination of the basis }
        \PY{c+c1}{\PYZsh{}    functions which is the maximum likelihood learned model.}
        
        \PY{k}{def} \PY{n+nf}{lsefit}\PY{p}{(}\PY{n}{x}\PY{p}{,} \PY{n}{t}\PY{p}{,} \PY{n}{params}\PY{p}{,} \PY{n}{func}\PY{p}{,} \PY{n}{mu}\PY{p}{)}\PY{p}{:} 
            \PY{n}{design\PYZus{}matrix} \PY{o}{=} \PY{n}{better\PYZus{}eval\PYZus{}basis}\PY{p}{(}\PY{n}{params} \PY{o}{=} \PY{n}{params}\PY{p}{,}
                                              \PY{n}{func} \PY{o}{=} \PY{n}{func}\PY{p}{,}
                                              \PY{n}{xeval} \PY{o}{=} \PY{n}{x}\PY{p}{)}
            
            \PY{n}{w\PYZus{}hat} \PY{o}{=} \PY{n}{lg}\PY{o}{.}\PY{n}{inv}\PY{p}{(}\PY{n}{np}\PY{o}{.}\PY{n}{dot}\PY{p}{(}\PY{n}{np}\PY{o}{.}\PY{n}{transpose}\PY{p}{(}\PY{n}{design\PYZus{}matrix}\PY{p}{)}\PY{p}{,} \PY{n}{design\PYZus{}matrix}\PY{p}{)} \PY{o}{+} 
                           \PY{n}{np}\PY{o}{.}\PY{n}{identity}\PY{p}{(}\PY{n}{design\PYZus{}matrix}\PY{o}{.}\PY{n}{shape}\PY{p}{[}\PY{l+m+mi}{1}\PY{p}{]}\PY{p}{)}\PY{o}{*}\PY{p}{(}\PY{l+m+mi}{1}\PY{o}{/}\PY{n}{mu}\PY{o}{*}\PY{o}{*}\PY{l+m+mi}{2}\PY{p}{)}\PY{p}{)}
            \PY{n}{w} \PY{o}{=} \PY{n}{np}\PY{o}{.}\PY{n}{dot}\PY{p}{(}\PY{n}{w\PYZus{}hat}\PY{p}{,} \PY{n}{np}\PY{o}{.}\PY{n}{dot}\PY{p}{(}\PY{n}{np}\PY{o}{.}\PY{n}{transpose}\PY{p}{(}\PY{n}{design\PYZus{}matrix}\PY{p}{)}\PY{p}{,} \PY{n}{t}\PY{p}{)}\PY{p}{)}
            \PY{k}{return} \PY{n}{w}
            
            
\end{Verbatim}


    \begin{Verbatim}[commandchars=\\\{\}]
{\color{incolor}In [{\color{incolor}3}]:} \PY{n}{lab\PYZus{}4\PYZus{}dat} \PY{o}{=} \PY{n}{scipy}\PY{o}{.}\PY{n}{loadmat}\PY{p}{(}\PY{l+s+s1}{\PYZsq{}}\PY{l+s+s1}{/Users/wyattmadden/Documents/school/}\PY{l+s+s1}{\PYZsq{}} \PY{o}{+} 
                                   \PY{l+s+s1}{\PYZsq{}}\PY{l+s+s1}{MSU/2020/spring/m508/lab\PYZus{}info/lab\PYZus{}4/simple.mat}\PY{l+s+s1}{\PYZsq{}}\PY{p}{,}
                                   \PY{n}{squeeze\PYZus{}me} \PY{o}{=} \PY{k+kc}{True}\PY{p}{)}
        
        \PY{n}{x} \PY{o}{=} \PY{n}{lab\PYZus{}4\PYZus{}dat}\PY{p}{[}\PY{l+s+s1}{\PYZsq{}}\PY{l+s+s1}{x}\PY{l+s+s1}{\PYZsq{}}\PY{p}{]}
        \PY{n}{t} \PY{o}{=} \PY{n}{lab\PYZus{}4\PYZus{}dat}\PY{p}{[}\PY{l+s+s1}{\PYZsq{}}\PY{l+s+s1}{t}\PY{l+s+s1}{\PYZsq{}}\PY{p}{]}
        
        \PY{n}{data\PYZus{}orig} \PY{o}{=} \PY{p}{\PYZob{}}\PY{l+s+s1}{\PYZsq{}}\PY{l+s+s1}{x}\PY{l+s+s1}{\PYZsq{}}\PY{p}{:} \PY{n}{x}\PY{p}{,}
                  \PY{l+s+s1}{\PYZsq{}}\PY{l+s+s1}{t}\PY{l+s+s1}{\PYZsq{}}\PY{p}{:} \PY{n}{t}\PY{p}{\PYZcb{}}
        
        \PY{n}{data\PYZus{}orig} \PY{o}{=} \PY{n}{pd}\PY{o}{.}\PY{n}{DataFrame}\PY{p}{(}\PY{n}{data\PYZus{}orig}\PY{p}{)}
\end{Verbatim}


    \section{3.2}\label{section}

    \begin{Verbatim}[commandchars=\\\{\}]
{\color{incolor}In [{\color{incolor}4}]:} \PY{c+c1}{\PYZsh{}function to automate fitting process}
        \PY{k}{def} \PY{n+nf}{df\PYZus{}of\PYZus{}preds}\PY{p}{(}\PY{n}{x}\PY{p}{,} \PY{n}{t}\PY{p}{,} \PY{n}{basis}\PY{p}{,} \PY{n}{func}\PY{p}{,} \PY{n}{mu}\PY{p}{,} \PY{n}{M}\PY{p}{,} \PY{n}{at}\PY{p}{)}\PY{p}{:}
            \PY{n}{fits} \PY{o}{=} \PY{n}{lsefit}\PY{p}{(}\PY{n}{x} \PY{o}{=} \PY{n}{x}\PY{p}{,}
                          \PY{n}{t} \PY{o}{=} \PY{n}{t}\PY{p}{,}
                          \PY{n}{params} \PY{o}{=} \PY{n}{basis}\PY{p}{(}\PY{l+m+mi}{0}\PY{p}{,} \PY{l+m+mi}{2}\PY{o}{*}\PY{n}{np}\PY{o}{.}\PY{n}{pi}\PY{p}{,} \PY{n}{M}\PY{p}{)}\PY{p}{,}
                          \PY{n}{func} \PY{o}{=} \PY{n}{func}\PY{p}{,} 
                          \PY{n}{mu} \PY{o}{=} \PY{n}{mu}\PY{p}{)}
            \PY{n}{preds\PYZus{}df} \PY{o}{=} \PY{p}{\PYZob{}}\PY{l+s+s1}{\PYZsq{}}\PY{l+s+s1}{fits}\PY{l+s+s1}{\PYZsq{}}\PY{p}{:} \PY{n}{np}\PY{o}{.}\PY{n}{dot}\PY{p}{(}\PY{n}{fits}\PY{p}{,} 
                                          \PY{n}{np}\PY{o}{.}\PY{n}{transpose}\PY{p}{(}\PY{n}{better\PYZus{}eval\PYZus{}basis}\PY{p}{(}\PY{n}{basis}\PY{p}{(}\PY{l+m+mi}{0}\PY{p}{,} 
                                                                  \PY{l+m+mi}{2}\PY{o}{*}\PY{n}{np}\PY{o}{.}\PY{n}{pi}\PY{p}{,} 
                                                                  \PY{n}{M}\PY{p}{)}\PY{p}{,} 
                                                                         \PY{n}{func}\PY{p}{,} 
                                                                         \PY{n}{at}\PY{p}{)}\PY{p}{)}\PY{p}{)}\PY{p}{,}
                        \PY{l+s+s1}{\PYZsq{}}\PY{l+s+s1}{x}\PY{l+s+s1}{\PYZsq{}}\PY{p}{:} \PY{n}{at}\PY{p}{\PYZcb{}}
            \PY{n}{preds\PYZus{}df} \PY{o}{=} \PY{n}{pd}\PY{o}{.}\PY{n}{DataFrame}\PY{p}{(}\PY{n}{preds\PYZus{}df}\PY{p}{)}
            \PY{k}{return} \PY{n}{preds\PYZus{}df}
        
        \PY{c+c1}{\PYZsh{}function to automate plotting}
        \PY{k}{def} \PY{n+nf}{plot\PYZus{}preds\PYZus{}and\PYZus{}data}\PY{p}{(}\PY{n}{preds}\PY{p}{,} \PY{n}{data}\PY{p}{,} \PY{n}{title}\PY{p}{)}\PY{p}{:}
            \PY{n}{sns}\PY{o}{.}\PY{n}{lineplot}\PY{p}{(}\PY{n}{x} \PY{o}{=} \PY{l+s+s2}{\PYZdq{}}\PY{l+s+s2}{x}\PY{l+s+s2}{\PYZdq{}}\PY{p}{,} \PY{n}{y} \PY{o}{=} \PY{l+s+s2}{\PYZdq{}}\PY{l+s+s2}{fits}\PY{l+s+s2}{\PYZdq{}}\PY{p}{,} \PY{n}{data} \PY{o}{=} \PY{n}{preds}\PY{p}{)}
            \PY{n}{sns}\PY{o}{.}\PY{n}{scatterplot}\PY{p}{(}\PY{n}{x} \PY{o}{=} \PY{l+s+s2}{\PYZdq{}}\PY{l+s+s2}{x}\PY{l+s+s2}{\PYZdq{}}\PY{p}{,} \PY{n}{y} \PY{o}{=} \PY{l+s+s2}{\PYZdq{}}\PY{l+s+s2}{t}\PY{l+s+s2}{\PYZdq{}}\PY{p}{,} \PY{n}{data} \PY{o}{=} \PY{n}{data}\PY{p}{)}\PY{o}{.}\PY{n}{set}\PY{p}{(}\PY{n}{title} \PY{o}{=} \PY{n}{title}\PY{p}{)}
            
            
            
\end{Verbatim}


    \begin{Verbatim}[commandchars=\\\{\}]
{\color{incolor}In [{\color{incolor}21}]:} \PY{n}{zero\PYZus{}to\PYZus{}two\PYZus{}pi} \PY{o}{=} \PY{n}{np}\PY{o}{.}\PY{n}{arange}\PY{p}{(}\PY{l+m+mi}{0}\PY{p}{,} \PY{l+m+mi}{2}\PY{o}{*}\PY{n}{np}\PY{o}{.}\PY{n}{pi}\PY{p}{,} \PY{n}{np}\PY{o}{.}\PY{n}{pi}\PY{o}{/}\PY{l+m+mi}{100}\PY{p}{)}
         
         \PY{n}{fits} \PY{o}{=} \PY{n}{df\PYZus{}of\PYZus{}preds}\PY{p}{(}\PY{n}{x} \PY{o}{=} \PY{n}{x}\PY{p}{,} \PY{n}{t} \PY{o}{=} \PY{n}{t}\PY{p}{,} \PY{n}{basis} \PY{o}{=} \PY{n}{hat\PYZus{}basis}\PY{p}{,} 
                            \PY{n}{func} \PY{o}{=} \PY{n}{func\PYZus{}hat}\PY{p}{,} \PY{n}{mu} \PY{o}{=} \PY{l+m+mi}{10}\PY{o}{*}\PY{o}{*}\PY{l+m+mi}{5}\PY{p}{,} \PY{n}{M} \PY{o}{=} \PY{n+nb}{len}\PY{p}{(}\PY{n}{x}\PY{p}{)}\PY{p}{,}
                            \PY{n}{at} \PY{o}{=} \PY{n}{zero\PYZus{}to\PYZus{}two\PYZus{}pi}\PY{p}{)}
         
         \PY{n}{plot\PYZus{}preds\PYZus{}and\PYZus{}data}\PY{p}{(}\PY{n}{fits}\PY{p}{,} \PY{n}{data\PYZus{}orig}\PY{p}{,} \PY{l+s+s2}{\PYZdq{}}\PY{l+s+s2}{mu = 10,000}\PY{l+s+s2}{\PYZdq{}}\PY{p}{)}
\end{Verbatim}


    \begin{center}
    \adjustimage{max size={0.9\linewidth}{0.9\paperheight}}{output_6_0.png}
    \end{center}
    { \hspace*{\fill} \\}
    
    The fit of the hat basis function with mu of 10,000 is not a good
approximation of the data. It is too responsive to subsequent data
points, especially on the lower end of the x space.

    \section{3.3}\label{section}

    \begin{Verbatim}[commandchars=\\\{\}]
{\color{incolor}In [{\color{incolor}22}]:} \PY{n}{fits} \PY{o}{=} \PY{n}{df\PYZus{}of\PYZus{}preds}\PY{p}{(}\PY{n}{x}\PY{p}{,} \PY{n}{t}\PY{p}{,} \PY{n}{hat\PYZus{}basis}\PY{p}{,} \PY{n}{func\PYZus{}hat}\PY{p}{,} \PY{l+m+mi}{10}\PY{p}{,} \PY{n+nb}{len}\PY{p}{(}\PY{n}{x}\PY{p}{)}\PY{p}{,}
                           \PY{n}{at} \PY{o}{=} \PY{n}{zero\PYZus{}to\PYZus{}two\PYZus{}pi}\PY{p}{)}
         \PY{n}{plot\PYZus{}preds\PYZus{}and\PYZus{}data}\PY{p}{(}\PY{n}{fits}\PY{p}{,} \PY{n}{data\PYZus{}orig}\PY{p}{,} \PY{l+s+s2}{\PYZdq{}}\PY{l+s+s2}{mu = 10}\PY{l+s+s2}{\PYZdq{}}\PY{p}{)}
\end{Verbatim}


    \begin{center}
    \adjustimage{max size={0.9\linewidth}{0.9\paperheight}}{output_9_0.png}
    \end{center}
    { \hspace*{\fill} \\}
    
    \begin{Verbatim}[commandchars=\\\{\}]
{\color{incolor}In [{\color{incolor}23}]:} \PY{n}{fits} \PY{o}{=} \PY{n}{df\PYZus{}of\PYZus{}preds}\PY{p}{(}\PY{n}{x}\PY{p}{,} \PY{n}{t}\PY{p}{,} \PY{n}{hat\PYZus{}basis}\PY{p}{,} \PY{n}{func\PYZus{}hat}\PY{p}{,} \PY{l+m+mi}{1}\PY{p}{,} \PY{n+nb}{len}\PY{p}{(}\PY{n}{x}\PY{p}{)}\PY{p}{,}
                           \PY{n}{at} \PY{o}{=} \PY{n}{zero\PYZus{}to\PYZus{}two\PYZus{}pi}\PY{p}{)}
         \PY{n}{plot\PYZus{}preds\PYZus{}and\PYZus{}data}\PY{p}{(}\PY{n}{fits}\PY{p}{,} \PY{n}{data\PYZus{}orig}\PY{p}{,} \PY{l+s+s2}{\PYZdq{}}\PY{l+s+s2}{mu = 1}\PY{l+s+s2}{\PYZdq{}}\PY{p}{)}
\end{Verbatim}


    \begin{center}
    \adjustimage{max size={0.9\linewidth}{0.9\paperheight}}{output_10_0.png}
    \end{center}
    { \hspace*{\fill} \\}
    
    \section{3.4}\label{section}

    For higher mu values, the fit appears to be more determined by the data,
while the overall pattern of the fit is not changing across mu values.
For example, for mu = 10,000, the fit is overlaid across every data
point, indicating a likely overfit. For mu = 1, the fit is closer to a
flat line, and does not pick up on the sigmoidal shape of the data.

    \section{3.5}\label{section}

    \begin{Verbatim}[commandchars=\\\{\}]
{\color{incolor}In [{\color{incolor}24}]:} \PY{n}{fits} \PY{o}{=} \PY{n}{df\PYZus{}of\PYZus{}preds}\PY{p}{(}\PY{n}{x}\PY{p}{,} \PY{n}{t}\PY{p}{,} \PY{n}{gauss\PYZus{}basis}\PY{p}{,} \PY{n}{func\PYZus{}gauss}\PY{p}{,} \PY{l+m+mi}{10}\PY{o}{*}\PY{o}{*}\PY{l+m+mi}{5}\PY{p}{,} \PY{n+nb}{len}\PY{p}{(}\PY{n}{x}\PY{p}{)}\PY{p}{,}
                           \PY{n}{at} \PY{o}{=} \PY{n}{zero\PYZus{}to\PYZus{}two\PYZus{}pi}\PY{p}{)}
         \PY{n}{plot\PYZus{}preds\PYZus{}and\PYZus{}data}\PY{p}{(}\PY{n}{fits}\PY{p}{,} \PY{n}{data\PYZus{}orig}\PY{p}{,} \PY{l+s+s2}{\PYZdq{}}\PY{l+s+s2}{mu = 10,000}\PY{l+s+s2}{\PYZdq{}}\PY{p}{)}
\end{Verbatim}


    \begin{center}
    \adjustimage{max size={0.9\linewidth}{0.9\paperheight}}{output_14_0.png}
    \end{center}
    { \hspace*{\fill} \\}
    
    \section{3.6}\label{section}

    \begin{Verbatim}[commandchars=\\\{\}]
{\color{incolor}In [{\color{incolor}25}]:} \PY{n}{fits} \PY{o}{=} \PY{n}{df\PYZus{}of\PYZus{}preds}\PY{p}{(}\PY{n}{x}\PY{p}{,} \PY{n}{t}\PY{p}{,} \PY{n}{gauss\PYZus{}basis}\PY{p}{,} \PY{n}{func\PYZus{}gauss}\PY{p}{,} \PY{l+m+mi}{10}\PY{p}{,} \PY{n+nb}{len}\PY{p}{(}\PY{n}{x}\PY{p}{)}\PY{p}{,}
                           \PY{n}{at} \PY{o}{=} \PY{n}{zero\PYZus{}to\PYZus{}two\PYZus{}pi}\PY{p}{)}
         \PY{n}{plot\PYZus{}preds\PYZus{}and\PYZus{}data}\PY{p}{(}\PY{n}{fits}\PY{p}{,} \PY{n}{data\PYZus{}orig}\PY{p}{,} \PY{l+s+s2}{\PYZdq{}}\PY{l+s+s2}{mu = 10}\PY{l+s+s2}{\PYZdq{}}\PY{p}{)}
\end{Verbatim}


    \begin{center}
    \adjustimage{max size={0.9\linewidth}{0.9\paperheight}}{output_16_0.png}
    \end{center}
    { \hspace*{\fill} \\}
    
    \begin{Verbatim}[commandchars=\\\{\}]
{\color{incolor}In [{\color{incolor}26}]:} \PY{n}{fits} \PY{o}{=} \PY{n}{df\PYZus{}of\PYZus{}preds}\PY{p}{(}\PY{n}{x}\PY{p}{,} \PY{n}{t}\PY{p}{,} \PY{n}{gauss\PYZus{}basis}\PY{p}{,} \PY{n}{func\PYZus{}gauss}\PY{p}{,} \PY{l+m+mi}{1}\PY{p}{,} \PY{n+nb}{len}\PY{p}{(}\PY{n}{x}\PY{p}{)}\PY{p}{,}
                           \PY{n}{at} \PY{o}{=} \PY{n}{zero\PYZus{}to\PYZus{}two\PYZus{}pi}\PY{p}{)}
         \PY{n}{plot\PYZus{}preds\PYZus{}and\PYZus{}data}\PY{p}{(}\PY{n}{fits}\PY{p}{,} \PY{n}{data\PYZus{}orig}\PY{p}{,} \PY{l+s+s2}{\PYZdq{}}\PY{l+s+s2}{mu = 1}\PY{l+s+s2}{\PYZdq{}}\PY{p}{)}
\end{Verbatim}


    \begin{center}
    \adjustimage{max size={0.9\linewidth}{0.9\paperheight}}{output_17_0.png}
    \end{center}
    { \hspace*{\fill} \\}
    
    \section{3.7}\label{section}

    Mu can be interpretted fairly similarly as in the hat basis case. For mu
= 10,000, the fit is extremely responsive to the data, and responds
chaotically outside the range of the observed data. For mu = 1, the fit
does not appear to be responsive enough. Mu = 10 is just right. The main
difference between the gaussian basis and the hat basis, is the hat
basis results in a much choppier fit.

    \section{3.8}\label{section}

    \begin{Verbatim}[commandchars=\\\{\}]
{\color{incolor}In [{\color{incolor}12}]:} \PY{n}{lab\PYZus{}4\PYZus{}test} \PY{o}{=} \PY{n}{scipy}\PY{o}{.}\PY{n}{loadmat}\PY{p}{(}\PY{l+s+s1}{\PYZsq{}}\PY{l+s+s1}{/Users/wyattmadden/Documents/school/}\PY{l+s+s1}{\PYZsq{}} \PY{o}{+} 
                                    \PY{l+s+s1}{\PYZsq{}}\PY{l+s+s1}{MSU/2020/spring/m508/lab\PYZus{}info/lab\PYZus{}4/test.mat}\PY{l+s+s1}{\PYZsq{}}\PY{p}{,}
                                    \PY{n}{squeeze\PYZus{}me} \PY{o}{=} \PY{k+kc}{True}\PY{p}{)}
         
         \PY{n}{x\PYZus{}test} \PY{o}{=} \PY{n}{lab\PYZus{}4\PYZus{}test}\PY{p}{[}\PY{l+s+s1}{\PYZsq{}}\PY{l+s+s1}{test\PYZus{}x}\PY{l+s+s1}{\PYZsq{}}\PY{p}{]}
         \PY{n}{t\PYZus{}test} \PY{o}{=} \PY{n}{lab\PYZus{}4\PYZus{}test}\PY{p}{[}\PY{l+s+s1}{\PYZsq{}}\PY{l+s+s1}{test\PYZus{}t}\PY{l+s+s1}{\PYZsq{}}\PY{p}{]}
         
         \PY{n}{data\PYZus{}test} \PY{o}{=} \PY{p}{\PYZob{}}\PY{l+s+s1}{\PYZsq{}}\PY{l+s+s1}{x}\PY{l+s+s1}{\PYZsq{}}\PY{p}{:} \PY{n}{x\PYZus{}test}\PY{p}{,}
                      \PY{l+s+s1}{\PYZsq{}}\PY{l+s+s1}{t}\PY{l+s+s1}{\PYZsq{}}\PY{p}{:} \PY{n}{t\PYZus{}test}\PY{p}{\PYZcb{}}
         
         \PY{n}{data\PYZus{}test} \PY{o}{=} \PY{n}{pd}\PY{o}{.}\PY{n}{DataFrame}\PY{p}{(}\PY{n}{data\PYZus{}test}\PY{p}{)}
         \PY{n}{sq\PYZus{}errors} \PY{o}{=} \PY{n}{np}\PY{o}{.}\PY{n}{empty}\PY{p}{(}\PY{l+m+mi}{100}\PY{p}{)}
         
         \PY{k}{for} \PY{n}{i} \PY{o+ow}{in} \PY{n+nb}{range}\PY{p}{(}\PY{l+m+mi}{1}\PY{p}{,} \PY{l+m+mi}{101}\PY{p}{)}\PY{p}{:}
             \PY{n}{fits} \PY{o}{=} \PY{n}{df\PYZus{}of\PYZus{}preds}\PY{p}{(}\PY{n}{x}\PY{p}{,} \PY{n}{t}\PY{p}{,} \PY{n}{gauss\PYZus{}basis}\PY{p}{,} \PY{n}{func\PYZus{}gauss}\PY{p}{,} \PY{n}{i}\PY{p}{,} \PY{l+m+mi}{10}\PY{p}{,}
                           \PY{n}{at} \PY{o}{=} \PY{n}{x\PYZus{}test}\PY{p}{)}
             \PY{n}{sq\PYZus{}errors}\PY{p}{[}\PY{n}{i}\PY{o}{\PYZhy{}}\PY{l+m+mi}{1}\PY{p}{]} \PY{o}{=} \PY{n}{np}\PY{o}{.}\PY{n}{sum}\PY{p}{(}\PY{p}{(}\PY{n}{fits}\PY{p}{[}\PY{l+s+s1}{\PYZsq{}}\PY{l+s+s1}{fits}\PY{l+s+s1}{\PYZsq{}}\PY{p}{]} \PY{o}{\PYZhy{}} \PY{n}{t\PYZus{}test}\PY{p}{)}\PY{o}{*}\PY{o}{*}\PY{l+m+mi}{2}\PY{p}{)}
         
         
         \PY{n}{sq\PYZus{}errors\PYZus{}and\PYZus{}mus} \PY{o}{=} \PY{p}{\PYZob{}}\PY{l+s+s1}{\PYZsq{}}\PY{l+s+s1}{mu}\PY{l+s+s1}{\PYZsq{}}\PY{p}{:} \PY{n+nb}{range}\PY{p}{(}\PY{l+m+mi}{1}\PY{p}{,} \PY{l+m+mi}{101}\PY{p}{)}\PY{p}{,}
                             \PY{l+s+s1}{\PYZsq{}}\PY{l+s+s1}{total\PYZus{}sq\PYZus{}error}\PY{l+s+s1}{\PYZsq{}}\PY{p}{:} \PY{n}{sq\PYZus{}errors}\PY{p}{\PYZcb{}}
         \PY{n}{sq\PYZus{}errors\PYZus{}and\PYZus{}mus} \PY{o}{=} \PY{n}{pd}\PY{o}{.}\PY{n}{DataFrame}\PY{p}{(}\PY{n}{sq\PYZus{}errors\PYZus{}and\PYZus{}mus}\PY{p}{)}
         
         \PY{n}{sns}\PY{o}{.}\PY{n}{scatterplot}\PY{p}{(}\PY{n}{x} \PY{o}{=} \PY{l+s+s2}{\PYZdq{}}\PY{l+s+s2}{mu}\PY{l+s+s2}{\PYZdq{}}\PY{p}{,} 
                         \PY{n}{y} \PY{o}{=} \PY{l+s+s2}{\PYZdq{}}\PY{l+s+s2}{total\PYZus{}sq\PYZus{}error}\PY{l+s+s2}{\PYZdq{}}\PY{p}{,} 
                         \PY{n}{data} \PY{o}{=} \PY{n}{sq\PYZus{}errors\PYZus{}and\PYZus{}mus}\PY{p}{)}\PY{o}{.}\PY{n}{set}\PY{p}{(}\PY{n}{title} \PY{o}{=} \PY{l+s+s2}{\PYZdq{}}\PY{l+s+s2}{Sq Error, Mu 1\PYZhy{}100}\PY{l+s+s2}{\PYZdq{}}\PY{p}{)}
\end{Verbatim}


\begin{Verbatim}[commandchars=\\\{\}]
{\color{outcolor}Out[{\color{outcolor}12}]:} [Text(0.5,1,'Sq Error, Mu 1-100')]
\end{Verbatim}
            
    \begin{center}
    \adjustimage{max size={0.9\linewidth}{0.9\paperheight}}{output_21_1.png}
    \end{center}
    { \hspace*{\fill} \\}
    
    \section{3.9}\label{section}

    \begin{Verbatim}[commandchars=\\\{\}]
{\color{incolor}In [{\color{incolor}34}]:} \PY{n}{sq\PYZus{}errors\PYZus{}and\PYZus{}mus}\PY{p}{[}\PY{l+s+s2}{\PYZdq{}}\PY{l+s+s2}{mu}\PY{l+s+s2}{\PYZdq{}}\PY{p}{]}\PY{p}{[}\PY{n}{np}\PY{o}{.}\PY{n}{argmin}\PY{p}{(}\PY{n}{sq\PYZus{}errors}\PY{p}{)}\PY{p}{]}
\end{Verbatim}


\begin{Verbatim}[commandchars=\\\{\}]
{\color{outcolor}Out[{\color{outcolor}34}]:} 16
\end{Verbatim}
            
    According the the above calculations, mu of 16 (for an integer value)
performs best on the data. This is because a mu of 16 results in the
lowest test error, ie. the best tradeoff between over and under fitting.

    \section{3.10}\label{section}

    \begin{Verbatim}[commandchars=\\\{\}]
{\color{incolor}In [{\color{incolor}19}]:} \PY{n}{m\PYZus{}sq\PYZus{}errors} \PY{o}{=} \PY{n}{np}\PY{o}{.}\PY{n}{empty}\PY{p}{(}\PY{l+m+mi}{99}\PY{p}{)}
         
         \PY{k}{for} \PY{n}{i} \PY{o+ow}{in} \PY{n+nb}{range}\PY{p}{(}\PY{l+m+mi}{2}\PY{p}{,} \PY{l+m+mi}{101}\PY{p}{)}\PY{p}{:}
             \PY{n}{fits} \PY{o}{=} \PY{n}{df\PYZus{}of\PYZus{}preds}\PY{p}{(}\PY{n}{x}\PY{p}{,} \PY{n}{t}\PY{p}{,} \PY{n}{gauss\PYZus{}basis}\PY{p}{,} \PY{n}{func\PYZus{}gauss}\PY{p}{,} \PY{l+m+mi}{13}\PY{p}{,} \PY{n}{i}\PY{p}{,}
                           \PY{n}{at} \PY{o}{=} \PY{n}{x\PYZus{}test}\PY{p}{)}
             \PY{n}{m\PYZus{}sq\PYZus{}errors}\PY{p}{[}\PY{n}{i}\PY{o}{\PYZhy{}}\PY{l+m+mi}{2}\PY{p}{]} \PY{o}{=} \PY{n}{np}\PY{o}{.}\PY{n}{sum}\PY{p}{(}\PY{p}{(}\PY{n}{fits}\PY{p}{[}\PY{l+s+s1}{\PYZsq{}}\PY{l+s+s1}{fits}\PY{l+s+s1}{\PYZsq{}}\PY{p}{]} \PY{o}{\PYZhy{}} \PY{n}{t\PYZus{}test}\PY{p}{)}\PY{o}{*}\PY{o}{*}\PY{l+m+mi}{2}\PY{p}{)}
         
         
         \PY{n}{m\PYZus{}sq\PYZus{}errors\PYZus{}and\PYZus{}mus} \PY{o}{=} \PY{p}{\PYZob{}}\PY{l+s+s1}{\PYZsq{}}\PY{l+s+s1}{M}\PY{l+s+s1}{\PYZsq{}}\PY{p}{:} \PY{n+nb}{range}\PY{p}{(}\PY{l+m+mi}{2}\PY{p}{,} \PY{l+m+mi}{101}\PY{p}{)}\PY{p}{,}
                             \PY{l+s+s1}{\PYZsq{}}\PY{l+s+s1}{total\PYZus{}sq\PYZus{}error}\PY{l+s+s1}{\PYZsq{}}\PY{p}{:} \PY{n}{m\PYZus{}sq\PYZus{}errors}\PY{p}{\PYZcb{}}
         \PY{n}{m\PYZus{}sq\PYZus{}errors\PYZus{}and\PYZus{}mus} \PY{o}{=} \PY{n}{pd}\PY{o}{.}\PY{n}{DataFrame}\PY{p}{(}\PY{n}{m\PYZus{}sq\PYZus{}errors\PYZus{}and\PYZus{}mus}\PY{p}{)}
         
         \PY{n}{sns}\PY{o}{.}\PY{n}{scatterplot}\PY{p}{(}\PY{n}{x} \PY{o}{=} \PY{l+s+s2}{\PYZdq{}}\PY{l+s+s2}{M}\PY{l+s+s2}{\PYZdq{}}\PY{p}{,} 
                         \PY{n}{y} \PY{o}{=} \PY{l+s+s2}{\PYZdq{}}\PY{l+s+s2}{total\PYZus{}sq\PYZus{}error}\PY{l+s+s2}{\PYZdq{}}\PY{p}{,} 
                         \PY{n}{data} \PY{o}{=} \PY{n}{m\PYZus{}sq\PYZus{}errors\PYZus{}and\PYZus{}mus}\PY{p}{)}\PY{o}{.}\PY{n}{set}\PY{p}{(}\PY{n}{title} \PY{o}{=} \PY{l+s+s2}{\PYZdq{}}\PY{l+s+s2}{Sq Error, M 1\PYZhy{}100}\PY{l+s+s2}{\PYZdq{}}\PY{p}{)}
\end{Verbatim}


\begin{Verbatim}[commandchars=\\\{\}]
{\color{outcolor}Out[{\color{outcolor}19}]:} [Text(0.5,1,'Sq Error, M 1-100')]
\end{Verbatim}
            
    \begin{center}
    \adjustimage{max size={0.9\linewidth}{0.9\paperheight}}{output_26_1.png}
    \end{center}
    { \hspace*{\fill} \\}
    
    \begin{Verbatim}[commandchars=\\\{\}]
{\color{incolor}In [{\color{incolor}36}]:} \PY{n}{m\PYZus{}sq\PYZus{}errors\PYZus{}and\PYZus{}mus}\PY{p}{[}\PY{l+s+s2}{\PYZdq{}}\PY{l+s+s2}{M}\PY{l+s+s2}{\PYZdq{}}\PY{p}{]}\PY{p}{[}\PY{n}{np}\PY{o}{.}\PY{n}{argmin}\PY{p}{(}\PY{n}{m\PYZus{}sq\PYZus{}errors}\PY{p}{)}\PY{p}{]}
\end{Verbatim}


\begin{Verbatim}[commandchars=\\\{\}]
{\color{outcolor}Out[{\color{outcolor}36}]:} 10
\end{Verbatim}
            
    Following similar arguments as in the mu assessement, an M of 10
provides the best tradeoff between over and under fitting.


    % Add a bibliography block to the postdoc
    
    
    
    \end{document}
